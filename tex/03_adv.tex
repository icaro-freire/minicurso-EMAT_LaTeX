For those wanting to adjust the margin sizes, or the \texttt{fancyhdr} layout, there are a few comments that could be made here.
\section{Page Dimensions}
\textit{NotesTeX} relies on the \texttt{geometry} package to set its dimensions. The associated code is the deceptively simple chunk of code given by
\begin{verbatim}
  \geometry{paperheight=845pt,paperwidth=597pt,
        marginparsep=.02\paperwidth,marginparwidth=.23\paperwidth,
        hoffset=-1in, voffset=-1in, headheight=.02\paperheight,
        headsep=.03\paperheight,footskip=20pt,
        textheight=.84\paperheight,textwidth=.64\paperwidth}.
\end{verbatim}
Ignoring most of the arguments, the \texttt{$\backslash$paperheight} and \texttt{$\backslash$paperwidth} are set to be the standard $8\times11$ inches but in \texttt{pt} format instead. All other options, with the exception of \texttt{$\backslash$hoffset} and \texttt{$\backslash$voffset}, inherit fractions of \texttt{$\backslash$paperheight} and \texttt{$\backslash$paperwidth}, the most important being \texttt{$\backslash$marginparwidth}. Increasing \texttt{$\backslash$marginparwidth} causes the margin to bleed off of the right side of the page. In order to increase, one \textbf{must} decrease the \texttt{$\backslash$textwidth} accordingly.


\section{\texttt{Fancyhdr} Layout}
As mentioned before, \texttt{fancyhdr} is overridden on the title page, the contents page, and the \texttt{$\backslash$part} page, and sets the header for all other pages through the code
\begin{verbatim}
\pagestyle{fancy}%
\newlength{\offset}%
\setlength{\offset}{\marginparwidth + \marginparsep}%
\renewcommand{\sectionmark}[1]{\markboth{#1}{}}%
\renewcommand{\subsectionmark}[1]{\markright{#1}{}}%

\fancypagestyle{fancynotes}{%
  \fancyhf{}%
  \fancyheadoffset[rh]{\offset}%
  \renewcommand{\headrulewidth}{0pt}%
  \fancyhead[L]{\textsc{\leftmark}}%
  \fancyhead[R]{\footnotesize \textit{\rightmark}~~~~ \thepage}%
}%
\end{verbatim}
The header style is set so that it spans the width of the entire page as opposed to just the \texttt{$\backslash$textwidth} through the line \texttt{$\backslash$fancyheadoffset[rh]\{$\backslash$myoddoffset\}}. The \texttt{$\backslash$sectionmark} and \texttt{$\backslash$subsectionmark} are set up so that the \texttt{section} appears on the left and all \texttt{subsections} appear on the right along with the page number, and this is given in the last two lines of code.