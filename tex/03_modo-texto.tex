\section{Texto e Código} %------------------------------------------------------
\label{sec:textoCodigo}

\begin{margintable}\vspace{.8in}\footnotesize
  \caption{Sumário da \textsc{part III}}
  \medskip
  \begin{tabularx}{\marginparwidth}{|X}
    \textbf{\sffamily \textcolor{azulUFRB}{Seção}~\ref{sec:textoCodigo}}.    {\sffamily Texto e Código} \\
  \end{tabularx}
\end{margintable}

A escrita do código em \LaTeX\ possui algumas peculiaridades.
Muitas delas, dependem de subjetividade e, portanto, fica à critério de cada um.
Vou falar apenas a minha experiência.

Quando iniciei no \LaTeX\ fazia códigos na mesma linha, aglutinando comandos.
Não me preocupava se num futuro eu teria dificuldade em entender o que escrevi, 
nem do trabalho que é encontrar um erro num código bagunçado!
Existem boas práticas e, desde cedo, aconselho vocês seguirem.\nota{
  Complementando essa ideia de \textsf{boas práticas} na escrita do código, o 
  texto 
  \hrefA
  {
    https://ctan.dcc.uchile.cl/info/l2tabu/english/l2tabuen.pdf
  }
  {
    \itshape An essential guide to \hologo{LaTeX2e} usage,
  }
  traz uma boa noção de comandos e pacotes obsoletos.
  É uma boa prática, sempre que possível, usar comandos mais novos e que, muitas 
  vezes, foram criados para resolver problemas de anteriores!
}

Na realidade, já estamos fazendo isso!
De fato, definimos uma estrutura de diretórios e organização dos arquivos 
\texttt{.tex}; indicamos como nomear arquivos; usamos comentários para separar 
partes do código; etc. 
Mas, a escrita, em si, deve comtemplar alguns aspectos também!
Ao meu ver, pense que todo o seu texto seja um código!
Uma boa prática, então, é manter cada linha de seu texto com, no máximo, 80 
caracteres.
Além disso, eu costumo colocar cada parágrafo em uma linha de código.
Isso é possível, pois o \LaTeX\ não considera mais de um espaçamento entre 
palavras!
Um novo parágrafo só é formado de você deixar uma linha em branco entre ele e o
anterior.

Por exemplo:

\begin{tcblisting}{title= Espaçamento entre palavras}
  Note         que, mesmo        eu    escrevendo
  espaçadamente, isso não    influencia no
  resultado!
  Mesmo dando ``Enter'' \ldots
  
  Parágrafo? Só se houver uma linha em branco entre eles!
\end{tcblisting}

Os espaçamentos \textsf{horizontais} podem ser obtidos com o comando 
\Verb|\hspace{x cm}|, onde $x$ representa o comprimento desejado.\mn{
  Merece destaque o comando \Verb|\hfill|. 
  Ele espaça, horizontalmente, duas palavras até o final da linha em curso.
}
Os \textsf{verticais} seguem a mesma ideia: \Verb|\vspace{}|.
É interessante notar que, se for usado números negativos, é feita uma contração 
no espaçamento. 

\begin{tcblisting}{title= Espaçamentos, listing side text}
Meu \hspace{2cm} Deus!\\
Meu \hspace{-0.5cm} Deus!\\
\vspace*{0.7cm}\\
Teste para vertical
\end{tcblisting}  