\section{Motivation}\label{sec:motivation}
\begin{margintable}\vspace{.8in}\footnotesize
  \begin{tabularx}{\marginparwidth}{|X}
  Section~\ref{sec:motivation}. Motivation\\
  Section~\ref{sec:reqpackages}. Required Packages\\
  Section~\ref{sec:license}. Margins\\
  \end{tabularx}
\end{margintable}
During my year as a Part III student at Cambridge, I realized that my theoretical physics professors, namely David Tong and David Skinner, would use the \texttt{jhep} format to typeset the notes for their classes. As the year went on, I started typesetting my personal notes during class and realized that the \texttt{jhep} format, while great for publications and lecture notes in general, was lacking a few small but useful features.

I came across James P. Sethna's wonderful text on statistical mechanics, published by the Oxford University Press, and loved the formatting of the OUP. Sadly the OUP does not have a publicly released LaTeX .sty file for their content, and while the Memoir class and the Tufte style packages provide extensive functionality, I needed something slightly different and a package that was more readily modifiable. Enter \textit{NotesTeX}.

The result of this year long work, from 2016-2017, is the package now known as \textit{NotesTeX.} The purpose of this package was to consolidate all these changes that I slowly incorporated into the original \texttt{jhep} format, and to provide stable support for commonly used physics and mathematics environments. I sincerely hope that you enjoy the package!


\section{Required Packages}\label{sec:reqpackages}
For \textit{NotesTeX,} the following packages are required
\begin{center}
  \texttt{marginnote, sidenotes, fancyhdr, titlesec, geometry, and tcolorbox.}
\end{center}
The roles of each of these packages will be discussed in Part~\ref{Part:Modification}. However, for a brief summary, the \texttt{marginnote}, \texttt{sidenote}, \texttt{titlesec}, and \texttt{tcolorbox} packages are used in creating the \texttt{$\backslash$part} environment, the package \texttt{geometry} is used globally to set the page width, page height, and margin width, and finally, \texttt{fancyhdr}, which is overridden on the title page, the contents page, and the \texttt{$\backslash$part} page, sets the header for the body.

\section{License}\label{sec:license}
Aditya Dhumuntarao does not own the copyright to the original package, \texttt{jheppub.sty}. All modification have been approved by the Jhep Editorial committee, and permission has been attributed to Aditya to distribute freely the modified version of \texttt{jheppub.sty}, known as \texttt{NotesTeX.sty}.

This work may be distributed and/or modified under the conditions of the LaTeX Project Public License, either version 1.3 of this license or (at your option) any later version. The latest version of this license is found \href{http://www.latex-project.org/lppl.txt}{here}, and version 1.3 or later is part of all distributions of LaTeX version 2005/12/01 or later. The current maintainer of this work is Aditya Dhumuntarao.\footnote{Please contact me at my email if you have any questions or comments.}

\newpage