\section{Uma Linguagem} %------------------------------------------------------
\label{sec:aprendendo}

\begin{margintable}\vspace{.8in}\footnotesize
  \caption{Sumário da \textsc{Part II}}
  \medskip
  \begin{tabularx}{\marginparwidth}{|X}
    \textbf{\sffamily \textcolor{azulUFRB}{Seção}~\ref{sec:aprendendo}}.    {\sffamily Uma Linguagem} \\
  \end{tabularx}
\end{margintable}

A aquisição de uma linguagem não é algo imediato, para a maioria das pessoas; e,
o \LaTeX{} é uma linguagem (de marcação).
Isso implica que há uma \textsf{estrutura} e \textsf{simbologia} características.

Vimos na Subseção~\ref{subsec:latex-word} que o \hologo{LaTeX} é uma linguagem 
que envolve \textit{códigos} e \textit{texto}.
Sua estruturação é diferente de sistemas \textsf{WYSYWYG} (como o Word) e, à
primeira vista, demanda um ``esforço técnico'' maior no início do aprendizado, 
mesmo para documentos simples.
Mas, à medida que a complexidade do documento aumenta, o esforço empregado ao 
usar o \LaTeX{} é menor, quando comparado à sistemas \textsf{WYSYWYG}, para 
produzir o mesmo documento.
A Figura~\ref{fig:latex-vs-word} mostra um vislumbre dessa ideia, embora seja
apenas hipotética.

\begin{figure}[!htbp]
  \centering
  \caption{Complexidade do Documento vs Esforço empregado}
  \label{fig:latex-vs-word}
  \medskip
  \includegraphics[width = 0.6\linewidth]{figs/latex-vs-word.png}
  
  {\small \textbf{Fonte:} \url{https://latexcfp.wordpress.com/}}
\end{figure}

\begin{marginfigure}
  \includegraphics[width = \linewidth]{piada-latex.png}
  
  {
    \sffamily
    \textit{Seu artigo não faz nehum sentido para 
    mim, mas é a coisa mais linda que eu já vi!}
    (Tradução livre e suavizada \emoji{grinning-face-with-sweat})
  }
  
  {\textsf{\textbf{Fonte:}} \href{http://nvisnjic.com/2015/01/13/mathjax-magic.html}{The magic of LaTeX.}}
\end{marginfigure}

Além disso, é notável a beleza da composição tipográfica final, em especial (mas
não necessariamente), quando o documento envolve equações matemáticas.

\subsection{Um pouco sobre a estruturação do LaTeX} %--------------------------

Antes de aprendermos os símbolos pertinentes dessa linguagem, é interessante 
conhecermos um pouco de sua estruturação.

Um comando no \LaTeX{} sempre começa com uma barra invertida (``\textbackslash'').
Tal comando pode está inserido em \textsf{modo texto} ou em \textsf{modo matemático}.\mn{
  Existe toda uma sintaxe para o \textsf{modo matemático}, que vamos deixar para
  abordar em outra seção.
}
Por exemplo, quando escrevemos \verb|\LaTeX|, com o ``L'', ``T'' e ``X'' maiúsculos, 
o resultado depois da compilação é: \LaTeX.
Mas, se digitarmos \verb|\Latex|, uma mensagem de erro será mostrada, visto que 
esse comando não vem por padrão definido.

Tal exemplo já nos mostra uma outra característica da estruturação da linguagem 
que estamos conhecendo: ela é \textit{case sensitive}, ou seja, é sensível à
letras maiúsculas e minúsculas.
O comando \verb|\LaTeX| é difernete do comando \verb|\latex|, que é difernte do
comando \verb|\Latex|.
\mn{
  \texttt{\textbackslash LaTeX} $\neq$ \texttt{\textbackslash Latex} $\neq$ \texttt{\textbackslash latex}
}

\begin{atencao}{Atenção!}{\exclamacao}
  Inclusive, essa é uma das grandes fontes de erro de quem está começando no 
  \LaTeX: a digitação de um comando de forma errada é muito comum!
\end{atencao}

Também é importante ressaltar que comandos podem vir acompanhados de 
\textbf{\textsf{argumentos obrigatórios}}, que colocamos entre \textsf{chaves} 
(``\{ \ldots \}''); ou de \textbf{\textsf{argumentos opcionais}}, que colacamos 
entre \textsf{colchetes} (``[\ldots]'').

Por exemeplo, o comando \verb|\documentclass| define a \textsf{classe} do 
documento.
O argumento da classe é obrigatório, ou seja, precisamos ``indicar'' ao \LaTeX{}
se estamos usando um \textsf{artigo} (\textit{article}); um \textsf{livro} 
(\textit{book}); um relatório (\textit{report}); uma apresentação (\textit{beamer});
etc.
Além disso, existem vários argumentos opcionais, dentre os quais descato: o 
tamanho da fonte, que por padrão é \texttt{10pt}\mn{ 
  são três opções possíveis nas classes \textit{standard}: \texttt{10pt}, 
  \texttt{11pt} e \texttt{12pt}.
}
; o formato do papel, por exemplo, \texttt{A4}; e a opção de \textit{layout} para
estruturar as margens à impressão em frente e verso: \texttt{twoside}.
Veja como ficaria o comando para a classe \texttt{article}; com tamanho da fonte 
de \texttt{12pt}; definida para um formato de paper \texttt{A4}; e, com a opção 
de imprimir em frente e verso:\mn{
  Note que o comando opcional \texttt{twoside} não ``diz'' para a sua impressora 
  imprimir em ``frente e verso''.
  Apenas configura as margens das páginas para essa opção.
}

\tcboxC{
  \ttfamily \textbackslash documentclass[12pt, a4paper, twoside]\{article\}
}

Notem que uma \textsf{classe} possui características instrínsecas: um artigo 
não possui capítulos, mas seções; o que é diferente de um livro, que possui 
essas duas estruturas, por exemplo.
Existem \textsf{classes} que já vem, por padrão, nas distribuições \TeX{} que 
conhecemos e inúmeras outras implementadas pela comunidade de usuários do \LaTeX.

A Tabela~\ref{tab:classes-nativas}, mostra algumas dessas \textsf{classes} 
\textsf{nativas} (também chamadas de \textsf{padrão} ou \textit{standard}).

\begin{margintable}
  \centering
  \caption{Algumas das \textsf{classes} \textit{standard} do \LaTeX}
  \label{tab:classes-nativas}
  \begin{tabular}{lcl}
    \toprule
      \textbs{Classe}  && \textbs{Finalidade} \\
    \midrule
      \textit{article} && artigo     \\
      \textit{book}    && livros     \\
      \textit{report}  && relatórios \\
      \textit{letter}  && carta      \\
    \bottomrule  
  \end{tabular}  
\end{margintable}

\begin{atencao}{Atenção!}{\exclamacao}
  Por conta da limitação de nosso tempo, vamos tangenciar alguns aspectos da 
  linguagem \hologo{LaTeX}, usando apenas a classe \textit{article}.
\end{atencao}

Você já deve ter imaginado que existe uma infinidade de \textsf{classes} para o
\LaTeX, além das nativas.
De fato, há uma comunidade ativa que contribui com seus talentos e disponibilizam,
em sua grande maioria, de forma gratuita muitas funcionalidades/ferramentas que 
facilitam a vida de usuários do \LaTeX.
Dentre essas funcionalidades/ferramentas, estão as \textsf{classes}.
Mas, existem \textbf{pacotes}; aplicativos em java (como o \textsf{arara}); etc.
Geralmente, as \textsf{classes} modificam a estrutura geral de um documento.
Como vimos, um \textsf{artigo} possui uma estrutura diferente de um \textsf{livro},
por exemplo.
O que define essas estruturas são as \textsf{classes}.\en{%
  A extensão do arquivo para uma classe é \texttt{.cls}
}

\begin{atencao}{Atenção!}{\exclamacao}
  Cada classe possui sua sintaxe específica para alguns comandos, porém alguns 
  outros comandos são gerais e servem para a maior parte das classes. 
  Estudaremos apenas esses últimos.
  Caso você queira trabalhar com uma classe diferente da “nativa”, deve 
  verificar o manual para os detalhes.
\end{atencao}

Dentre as classes que não são nativas ao \hologo{LaTeX}, gostaria de destacar:
\nota{
  É bem engraçada, e realista, a descrição do nome abn\TeX:\\ 
  {
    \centering \textit{ABsurd Norms for TeX}.\\
  }
  Por que será, né? \emoji{hand-over-mouth}
}

\begin{description}
  \item[\textas{abn\TeX 2}] Se você pensa em escrever seu Trabalho de Conslusão 
                   de Curso, usandoo \LaTeX, você precisa ler a documentação e 
                   conhecer bem essa classe.
                   Ela, basicamente, fornece um modelo canônico (\textit{template})
                   para várias estruturas de cunho científico, dentro do padrão 
                   da \textsc{abnt} (Associação Brasileira de Normas Técnicas).
                   Você não precisará se preocupar com as configurações de todas
                   aquelas normas em seu texto!
                   Conhecendo os comandos da classe, você só se preocupará com 
                   o \textsf{conteúdo}, não com a \textsf{estrutura}.
                   Inclusive, estou desenvolvendo uma clase para monografia do 
                   curso de Licenciatura em Matemática da UFRB, baseada na classe
                   abn\TeX 2.
                   Ela está disponível lo \textit{link}: 
                   \hrefB{https://github.com/icaro-freire/tccUFRB}{\textsf{tccUFRB.cls}}
  \item[\textas{beamer}] Esta é uma classe para apresentações em \textit{slides} 
                   (chamado também de \textit{frame}, pelos usuários).
                   A qualidade das apresentações é surpreendente. 
                   Principalmente quando a apresentação envolve fórmulas 
                   matemáticas (não distorce como acontece por aí \ldots)!
                   Existem uma infinidade de \textit{templates} (oficiais ou não) 
                   disponíveis que você, certamente, encontrará um que atenda às 
                   suas necessidades.
                   O manual do \textsf{beamer} pode ser visto no \textit{link}:
                   \hrefB
                   {
                     http://tug.ctan.org/macros/latex/contrib/beamer/doc/beameruserguide.pdf
                   }
                   {
                     The \textsc{beamer} \textit{class}
                   }
                   \newline
                   Um \textit{template} de tema claro, extra oficial, que 
                   recomendo para \textsf{beamer} é o 
                   \hrefB{https://github.com/matze/mtheme}{Metropolis};
                   e, um escuro (\textit{dark}): 
                   \hrefB{https://github.com/junwei-wang/beamerthemeNord}{The Nord Beamer Theme.} \\
                   Um pequeno manual, em espanhol, que considero muito prático e 
                   eficiente (principalmente se você não quiser ler as 234~p do 
                   manual oficial) é: 
                   \hrefB{http://metodos.fam.cie.uva.es/~latex/apuntes/apuntes13.pdf}{Presentaciones en \LaTeX\ con Beamer.}
  \item[\textas{exam}] Essa é uma classe para confecção de listas de atividade.
                   Possui um ambiente para soluções; comandos para pontuação nas
                   questões; cria cabeçalhos e rodapés estilizados; etc.
                   Se você é professor e usa fórmulas em suas listas de 
                   atividades, recomendo fortemente ler a documentação dessa 
                   classe \textsf{exam.cls}.
                   Você encontrará o manual nesse \textit{link}:
                   \hrefB{https://ctan.dcc.uchile.cl/macros/latex/contrib/exam/examdoc.pdf}{Using the exam document class}.
                   \newline
                   Baseada nessa classe, desenvolvi (e estou no processo de 
                   atualização) a classe \hrefB{https://github.com/icaro-freire/ativmatUFRB}{ativmatUFRB.cls}.
                   Ela é uma classe não oficial para listas de atividades do 
                   curso de Licenciatura em Matemática, da UFRB.
                   Possui um cabeçalho estilizado com o logo da universidade, bem
                   como comandos e informações que considero úteis num documento
                   desse tipo.
\end{description}



