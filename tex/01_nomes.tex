\section{A Seiva da Árvore \ldots}\label{sec:seiva} %==========================

\begin{margintable}\vspace{.8in}\footnotesize
  \begin{tabularx}{\marginparwidth}{|X}
    Seção~\ref{sec:seiva}. A seiva da árvore\\
    Seção~\ref{sec:seiva}. A seiva da árvore\\
    Seção~\ref{sec:seiva}. A seiva da árvore\\
  \end{tabularx}
\end{margintable}

Lembro-me da primeira vez que vi o nome \LaTeX \ldots

Foi em uma chamada de um minicurso de alguma "semana de matemática" da 
universidade que fiz graduação.
O título era algo assim: "Introdução ao \LaTeX".

Eu estava a um semestre de concluir a graduação e pensei: 
\textit{
  "Rapaz \ldots acho que eles erraram esse minicurso. 
  Pra quê estudar látex em Matemática? 
  Isso está mais para Geografia."
}

Sim! 
Eu pensei que estavam falando daquela substância espessa e branca que sai de 
algumas plantas (seringueira, por exemplo)!

Então, vamos deixar as coisas claras: não estamos falando de látex, mas de \LaTeX.

Falando nisso, a palavra \TeX, foi idealizada como sendo a junção de três letras
gregas: $\tau\epsilon\chi$. \footnote{$\tau$ (tau), $\epsilon$ (épsilon) e $\chi$ (chi)}
Esse núcleo grego gera palavras como \textit{arte} ($\small \tau\epsilon\chi\nu\eta$) ou 
\textit{tecnologia} ($\small \tau\epsilon\chi\nu o \lambda o \gamma\iota\alpha$).
Daí vem o espírito da palavra \TeX, ou seja, unir uma arte (a tipografia) com 
a tecnologia (programação) para produzir documentos com uma beleza ímpar, 
tipograficamente profissional, adequando-se perfeitamente à escrita de objetos
matemáticos.

\begin{marginfigure}
  \centering
  \href
  {
    https://www.globo.com/GloboRural/0,6993,EEC1703411-1935,00.html
  }
  {
    \includegraphics[width = 0.75\linewidth]{seringueira.jpg}
  }
  \caption{Isso é látex, não \LaTeX}
\end{marginfigure}





