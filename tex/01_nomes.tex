\section{A Seiva da Árvore \ldots}\label{sec:seiva} %==========================

\begin{margintable}\vspace{.8in}\footnotesize
  \caption{Sumário da \textsc{Part I}} 
  \medskip  
  \begin{tabularx}{\marginparwidth}{|X}
    \textbf{\sffamily \textcolor{azulUFRB}{Seção}~\ref{sec:seiva}}.    {\sffamily A seiva da árvore}                  \\
    \textbf{\sffamily \textcolor{azulUFRB}{Seção}~\ref{sec:usando}}.   {\sffamily Sim\ldots mas como eu uso?}         \\
    \textbf{\sffamily \textcolor{azulUFRB}{Seção}~\ref{sec:overleaf}}. {\sffamily Overleaf: um editor online notável} \\
  \end{tabularx}
\end{margintable}

Lembro-me da primeira vez que vi o nome \LaTeX{} \ldots

Foi em uma chamada de um minicurso de alguma ``semana de matemática'' da 
universidade que fiz graduação.
O título era algo assim: ``Introdução ao \LaTeX''.

Eu estava a um semestre de concluir a graduação e pensei: 
\textit{
  ``Rapaz \ldots acho que eles erraram esse minicurso. 
  Pra quê estudar látex em Matemática? 
  Isso está mais para Geografia.''
}

Sim! 
Eu pensei que estavam falando daquela substância espessa e branca que sai de 
algumas plantas (seringueira, por exemplo)!

Então, vamos deixar as coisas claras: não estamos falando de látex, mas de \LaTeX.

\begin{atencao}{Como pronunciar corretamente?}{\exclamacao}
  Aliás, como se pronuncia a palavra \LaTeX? \\
  Existem, pelo menos, duas maneiras de pronunciarmos, corretamente, essa palavra:
  ``\textit{LeiTéc}'' ou ``\textit{LaTéc}'', ou seja, o som de \TeX{} (``Téc'') 
  é o mesmo que na palavra ``\textit{tec}nologia''. 
  Particularmente, adoto a segunda opção.
  Evite, por amor a Deus, falar ``\textit{Látecks}''. \emoji{face-with-rolling-eyes}
\end{atencao}

\begin{marginfigure}
  \centering
  \caption{Isso é látex, não \LaTeX}
  \medskip
  \href
  {
    https://www.globo.com/GloboRural/0,6993,EEC1703411-1935,00.html
  }
  {
    \includegraphics[width = 0.75\linewidth]{seringueira.jpg}
  }

  {\small \sffamily Fonte: Globo Rural}
\end{marginfigure}

Falando nisso, essa palavra \TeX{} não está destacada de forma aleatória! 
Ela foi idealizada como sendo a junção de três letras gregas: \foreignlanguage{greek}{τεχ}.
\footnote{%
\foreignlanguage{greek}{τ} (tau), \foreignlanguage{greek}{ε} (épsilon) e \foreignlanguage{greek}{χ} (chi)
}
Esse núcleo grego gera palavras como \textit{arte} (\foreignlanguage{greek}{τέχνη}) 
ou mesmo \textit{tecnologia} (\foreignlanguage{greek}{τεχνολογία}).
Daí vem o espírito da palavra \TeX: unir uma arte (a tipografia) com a tecnologia 
(programação) para produzir documentos com uma beleza realmente ímpar.

Na realidade, para falarmos com propriedade sobre \LaTeX, precisamos tangenciar 
o \TeX.

\subsection{Diferenciando os nomes} %------------------------------------------

Tudo começou quando \href{https://pt.wikipedia.org/wiki/Donald_Knuth}{Donald Ervin Knuth} 
queria qualidade nos elementos tipográficos de seus livros, principalmente na 
escrita matemática. 
Ele então criou, em meados da década dos anos 70, um sofisticado programa para a 
composição tipográfica de textos científicos e uma alternativa quase necessária 
para edição de textos com conteúdo matemático. 
Nasceu o \TeX. 
Em suas próprias palavras:

\begin{cita}
  \TeX{} é destinado para a criação de belos livros - e, especialmente, para os 
  livros que contêm grande quantidade de matemática.
\end{cita}

Todavia, ao que parece, essa linguagem de programação não era tão acessível 
\sout{a meros mortais como nós}, por conter diversos parâmetros relativos ao 
formato final do texto \ldots
Bom, vamos falar a verdade: dava muito trabalho para digitar o que você queria 
simplificar.
Isso porque o \TeX{} era formado por objetos denominados ``primitivos'' e toda 
estruturação do texto deveria ser feita por meio deles.
O próprio Knuth criou um conjunto de macros\nota{
  Uma \textit{macro} (abreviação para macroinstrução), em ciência da computação, 
  é uma regra ou padrão que especifica como uma certa “sequência de entrada” 
  deve ser mapeada para uma substituição de “sequência de saída” de acordo com 
  um procedimento definido.
}, ou seja, um mapeamento de sequências de comandos primitivos, frequentemente 
utilizados, para simplificar a escrita de comandos \TeX{} em seus livros.
A esse conjunto de macros ele deu o nome de \textit{Plain}~\TeX.

Obviamente, o \textit{Plain}~\TeX{} é um conjunto de macros bem simples e que 
poderia ser expandido.
E foi isso que aconteceu.

\subsection{O surgimento do \LaTeX} %------------------------------------------

\href{https://pt.wikipedia.org/wiki/Leslie_Lamport}{Leslie B. Lamport}, facilitou 
nossa vida! 
Por volta dos anos 80, ele criou um conjunto de macros para o \TeX, muito bem 
estruturado, e com ideias interessantes (classes e pacotes, por exemplo) que 
somaram substanciamente à causa do \TeX{}, formando assim o \LaTeX.
Inclusive, o ``La'', de \LaTeX{}; vem do ``La'', de \textit{\textbf{La}mport}.
\mn{
  $\text{\LaTeX} = \text{\textbf{La}mport} + \text{\TeX}$
}

Logo, nunca se esqueça disso:

\tcboxC{
  \large  \textsf{O} \LaTeX\ \textsf{veio para facilitar sua vida}!
}

\begin{figure}[!ht]
  \centering
  \includegraphics[width = 0.8\linewidth]{lamport-knuth}
  \caption{Os culpados!}
\end{figure}

\subsection{Mais e mais nomes: compiladores, interpretadores e formatos} %-----

Bom \ldots já sabemos que  \TeX{} e \LaTeX{} são coisas diferentes, entretanto,
interrelacionadas: o segundo é a forma mais utilizada, atualmente, para 
interagir com o primeiro.

Nesse ponto, seria interessante diferenciarmos \textit{engines} (motores/interpretadotes) 
de \textit{formats} (formatos).

O \TeX{} é um interpretador (\textit{engine}); o \LaTeX, um formato (\textit{format}).

\nota{
  Um outro formato que vimos é o \textit{Plain}~\TeX; e, atualmente, um formato 
  que se destaca é o \hrefA{https://en.wikipedia.org/wiki/ConTeXt}{\hologo{ConTeXt}}.
}

Os \textit{interpretadores} são os arquivos binários executáveis (ou seja, o 
programa em si); já os \textit{formatos} são macros (comandos ou instruções), 
baseadas em \TeX, que usamos para escrever nossos documentos (a grosso modo, são
linguagens ou atalhos para sequências de comandos ou estruturas em \TeX).

Não é difícil perceber que o interpretador \TeX{} é bem antigo e que outros tenham
surgido ao longo desses anos.
De fato, à época do \TeX, nem existia ainda arquivos com extensão  \texttt{.pdf} 
-- a saída dos documentos era em DVI.
\begin{marginfigure}
  \centering
  \href
  {
    https://www.lua.org/portugues.html
  }
  {
    \includegraphics[width = 0.6\linewidth]{logo_lua.png}
  }
  \caption{Poderosa linguagem de programação (clique na imagem para saber mais)}
  \label{fig:lua}
\end{marginfigure}NotesTeX.
Quando surgiu o PDF, um interpretador ficou bem conhecido: \hologo{pdfTeX} -- cuja 
saída poderia ser em DVI que poderia ser convertida em PDF. 
O \hologo{pdfTeX} talvez seja o mais usado dentre os interpretadores que vamos 
falar, pois já vem selecionado, por padrão, em muitos editores dedicados ao 
\LaTeX \footnote{%
  falaremos sobre editores para \LaTeX{} mais abaixo.}; 
as pessoas, simplesmente, não mudam a configuração padrão se realmente não 
for necessário.
Aliás, nosso sistema relaciona o \LaTeX{} com o \hologo{pdfTeX} usando um 
``atalho'', denominado \hologo{pdfLaTeX}.

Portanto, o \hologo{pdfLaTeX} (que geralmente é o nome que aparecerá em nossos 
editores) é o \textit{interpretador} pdf\TeX\ com o \textit{formato} \LaTeX. 

Atualmente, dois interpretadores que se destacam são \hologo{XeTeX} e \hologo{LuaTeX}.
Além de serem mais rápidos, trazem implementações notáveis como seleção de fontes 
do próprio sistema (o \hologo{pdfTeX} não faz isso) ou interação com a linguagem 
de programação (brasileira) Lua (veja a Figura~\ref{fig:lua}). 
\nota{
  De acordo com Joseph Wright, usamos a palavra ``compilação'' herdada da computação,
  mas, uma palavra mais adequada seria ``composição tipográfica'' \cite{learnlatex}.
}
A saída de cada um deles é o PDF, diretamente.

Em nossos sistemas, ao usarmos \hologo{XeTeX} com \LaTeX{} usamos o atalho 
\hologo{XeLaTeX}.
Da mesma forma, \hologo{LuaTeX} com \LaTeX{} é simplificado por \hologo{LuaLaTeX}.

Obviamente, nesse minicurso, estamos interessados apenas no formato \LaTeX.
Também, por mera preferência, usaremos como interpretador o \hologo{LuaTeX}.
Portanto, ``compilaremos'' \, nossos arquivos com \texttt{lualatex}.

Obtamos por usar \lualatex, pois é o sucessor natural do \pdflatex. 
\mn{
  Confira essa informação no site \hrefB{http://www.luatex.org/}{http://www.luatex.org/}
}

\begin{atencao}{Mudanças à vista \ldots}{\exclamacao}
  Isso pode mudar algumas coisas para quem é acostumado a usar \pdflatex, 
  mas explicaremos as diferenças ao longo desse minicurso.  
\end{atencao}

Geralmente chamamos esses ``atalhos'' de \textbf{compiladores}
\mn{
  Para mais detalhes, veja \textcite{lualatex-doc}.
}.

Veja a Tabela~\ref{tab:compiladores} para comparação entre \textit{compiladores}, 
\textit{engine} e \textit{format}.

\begin{table}[!htbp]
  \centering
  \begin{tabular}{llc}
    \toprule
      \textbs{Compilador}    & \textbs{Interpretador} & \textbs{Formato}  \\
    \midrule
      \texttt{tex}           & \hologo{TeX}           & \hologo{plainTeX} \\
      \texttt{pdftex}        & \hologo{pdfTeX}        & \hologo{plainTeX} \\
      \texttt{xetex}         & \hologo{XeTeX}         & \hologo{plainTeX} \\
      \texttt{latex}         & \hologo{pdfTeX}        & \LaTeX            \\
      \texttt{pdflatex}      & \hologo{pdfTeX}        & \LaTeX            \\
      \texttt{xelatex}       & \hologo{XeTeX}         & \LaTeX            \\
      \texttt{lualatex}      & \hologo{LuaTeX}        & \LaTeX            \\
      \texttt{texexec}       & \hologo{pdfTeX}        & \hologo{ConTeXt}  \\
      \texttt{texexec --xtx} & \hologo{XeTeX}         & \hologo{ConTeXt}  \\
      \texttt{context}       & \hologo{LuaTeX}        & \hologo{ConTeXt}  \\
    \bottomrule
  \end{tabular}
  \caption{Dando ``nome aos bois''}
  \label{tab:compiladores}
\end{table}

Como 
\begin{marginfigure}
  \centering
  \href
  {
    http://www.luatex.org/
  }
  {
    \includegraphics[width = 0.9 \linewidth]{logo_luatex.png}
  }
  \caption{
    Saiba mais sobre o \hologo{LuaTeX} acessando o link: 
    \href{http://www.luatex.org}{http://www.luatex.org}
  }
  \label{fig:luatex}
\end{marginfigure}
esse texto é uma \sout{tentativa de} introdução ao \LaTeX, não chegaremos 
nem perto do que gostaríamos de expor sobre essa simbiose entre uma linguagem de 
\textit{programação} (Lua) integrada harmoniosamente com uma linguagem de 
\textit{marcação} (\LaTeX).
Basicamente, o código do \TeX{} foi reescrito em Lua (uma linguagem de programação
com muita relevância internacional, desenvolvida por brasileiros, na PUC-RJ), 
formando assim o \hologo{LuaTeX}.
Houve, intencionalmente, a projeção para que esse \textit{interpretador} fosse 
compatível com versões anteriores do \hologo{pdfTeX}, tornando o \hologo{LuaTeX} 
seu substituto natural, visto que é mais rápido e moderno.

Para encerramos essa subseção, é importante destacar a estabilidade do \LaTeX.
Basicamente temos duas versões: uma antes de 1993, a saber \LaTeX~2.09; e,
a de 1994 até HOJE, a saber \hologo{LaTeX2e}.
Isso é interessante, pois os códigos são praticamente preservados ao longo do 
tempo: você poderia rodar um código em \LaTeX\ de 20 anos atrás sem muitos 
problemas.

Todavia, quando falamos em \LaTeX{}, hoje, estamos nos referindo às 
funcionalidades trazidas na versão \hologo{LaTeX2e}.

\begin{marginfigure}
  \centering
  \includegraphics[width = \linewidth]{logo_latex}
  \caption{
    Saiba mais sobre esse fantástico projeto no link:
    \href{https://www.latex-project.org}{https://www.latex-project.org}
  }
  \label{fig:latexiii}
\end{marginfigure}

As atualizações são constantes, mas sem muitas mudanças estruturais.
Todavia, está em desenvolvimento uma terceira versão do \LaTeX, a saber, 
\hologo{LaTeX3} (veja a Figura~\ref{fig:latexiii}).

%------------------------------------------------------------------------------
\subsection{Como colocar o \hologo{LaTeX} em meu computador: as Distribuições}
%------------------------------------------------------------------------------ 

Para instalar localmente (em nosso computador) os \textit{interpretadotes}, 
precisamos das \textbf{distribuições}.

As principais são:

\begin{table}[!htbp]
  \centering
  \begin{tabular}{lll}
    \toprule
      \textbs{Distribuições} & \textbs{Sistema} & \textbs{\textit{Download}/Instalação}\\
    \midrule
      \hologo{MiKTeX}   & \textrm{Windows ou GNU/Linux ou Mac OS} & \href{https://miktex.org/download}{https://miktex.org/download} \\ 
      \TeX{} Live       & \textrm{GNU/Linux ou Windows}           & \href{https://www.tug.org/texlive/acquire-netinstall.html}{https://www.tug.org/texlive/acquire-netinstall.html}\\
      Mac\TeX           & \textrm{Mac OS}                         & \href{https://www.tug.org/mactex/}{https://www.tug.org/mactex/}  \\
    \bottomrule
  \end{tabular}
\end{table}

A distribuição Mac\TeX{} contém todo o \TeX{} Live e adições específicas para 
Mac~OS.
\nota{
  Para GNU/Linux existem outras opções de instalação que economizam espaço.
  Tudo dependerá da necessidade de cada um:\\
  \texttt{texlive-latex-base}\\
  \texttt{texlive-latex-recommended}\\
  \texttt{texlive-pictures}\\
  \texttt{texlive-fonts-recommended}\\
  \texttt{texlive}\\
  \texttt{texlive-plain-generic}\\
  \texttt{texlive-latex-extra}\\
  Para mais informações veja essa discussão no \hrefA{https://tex.stackexchange.com/questions/245982/differences-between-texlive-packages-in-linux}{TeX SE}
}

O \TeX{} Live completo precisa de, aproximadamente, 5GB de espaço em disco.
Ele instala TODOS os pacotes disponíveis para \hologo{LaTeX}.
Como hoje em dia a capacidade de disco é relativamente grande, a instalação 
completa pode ser interessante, caso você não queira depender mais da internet 
para instalação de futuros pacotes.

Mas, se você quiser usar apenas o necessário e manter uma distribuição mínima em
seu computador, instalando futuros pacotes à medida que for precisando deles, 
talvez a distribuição \hologo{MiKTeX} seja a mais adequada.

\begin{atencao}{Atenção!}{\exclamacao}
  O processo de instalação depende do sistema operacioal e não será tratado nesse 
  texto, visto que usaremos distribuições ONLINE dos instrepretadores de \TeX{}.  
\end{atencao}

\section{Sim \ldots mas como eu uso?} %----------------------------------------
\label{sec:usando}

Certo \ldots já temos uma distribuição \TeX{}, e agora?
Basicamente, precisamos:

\begin{itemize}
  \item \textit{escrever}, usando o formato \hologo{LaTeX}, o que desejamos num 
    arquivo de texto com extensão \texttt{.tex};
  \item \textit{compilar}, ou seja, compor tipograficamente o arquivo \texttt{.tex}, 
    produzindo um arquivo \texttt{.pdf}; e, nesse ponto, usaremos o compilador 
    \texttt{lualatex};
  \item \textit{visualizar} o arquivo \texttt{.pdf}; nesse ponto, o visualizador
    de PDF é de escolha pessoal.
\end{itemize}

É uma boa prática salvar os arquivos \texttt{.tex} com nomes \textbf{sem acentuação}
ou \textbf{caracteres especiais} do teclado (\%, \$, *, etc.).
E, se o nome do arquivo for composto por mais de uma palavra, é aconselhável 
também \textbf{não deixar espaço entre elas}.

Por exemplo, suponha que você esteja escrevendo um artigo sobre \textit{números complexos}.
Seu arquivo principal não deve ser nomeado assim: 

\tcboxC{
  \texttt{Artigo! Números Complexos.tex}
}

Você deve retirar o \textit{ponto de exclamação} e o \textit{acento agudo}, bem 
como retirar os espaços entre as palavras.
Nesse último ponto, pode-se usar o \texttt{camelCase}, ou \texttt{snake\_case} 
ou separarar as \texttt{palavras-com-\textit{traço}}.
São aceitáveis qualquer das seguintes possibilidades (note que há uma preferência
por letras minúsculas):

%\begin{tcolorbox}
\tcboxC{
  \centering
  \begin{tabular}{l}
    \texttt{artigoNumerosComplexos.tex}\\
    \texttt{artigo-numeros-complexos.tex}\\
    \texttt{artigo\_numeros\_complexos.tex}\\
    \texttt{artigo\_numeros-complexos.tex}  
  \end{tabular}
}
%\end{tcolorbox}

Particularmente, prefiro essa última opção: onde se mistura o \textit{snake\_case}
com os traços.
Tento manter algum padrão antes do \textit{underline} e o nome do documento que 
estou escrevendo, separo por traços, caso seja necessário.
Essa abordagem pode ser interessante se você estiver trabalhando com um arquivo
principal (onde geralmente ocorre a compilação final) e muitos outros arquivos 
que serão incluídos no principal.

É uma forma de escrever \ldots
Há quem prefira escrever tudo em um único arquivo \texttt{.tex}!

Falaremos sobre manipulação de vários arquivos mais à frente, porém, apenas para
exemplificar a ideia, suponha que esse meu artigo (\texttt{artigo\_numeros-complexos.tex}) 
sobre números complexos seja composto de cinco seções:
\nota{
  Uma forma de nomear esses arquivos seria, respectivamente:\\
  \texttt{01\_intro-historica.tex}; \\
  \texttt{02\_representacoes.tex}; \\
  \texttt{03\_raizes.tex}; \\
  \texttt{04\_teo-fundamental.tex};\\
  \texttt{05\_conclusao.tex}.\\
  Notem que deixei a numeração no início, separando-a com \textit{underline} e 
  nomeei os arquivos de maneira que lembre-me a seção onde me encontro.
} 
introdução histórica; formas de representar um número complexo; raízes de um 
número complexo; o Teorema Fundamental da Álgebra; e, conclusões.
Então, desejamos escrever essas seções em arquivos separados e ``incluí-los'' no
arquivo principal, paulatinamente, por meio de compilações SOMENTE no arquivo
principal.

\begin{atencao}{Fica a dica!}{\exclamacao}
  Muitos nomeiam esses ``arquivos principais'', ou seja, onde ocorre a compilação, 
  de \texttt{main.tex} (ou \texttt{master.tex}), que significa, em inglês, 
  ``principal'', ``importante'', etc. (``senhor'', ``dominador'', etc., para \textit{master}).
\end{atencao}

Para esse minicurso, à jato, começaremos com um arquivo principal, nomeado por
``\texttt{main.tex}'', no qual escreveremos o conteúdo desejado inserindo as partes 
que o compõem, aos poucos.

\subsection{Como fazemos a ``compilação''?} %------------------------------------

Bom \ldots ainda aprenderemos como escrever em \hologo{LaTeX}, não se preocupem,
mas, antes, vamos aprender como compomos tipograficamente o texto.

Já vimos que precisamos escrever o conteúdo do texto em um arquivo de extensão
\texttt{.tex} e compilarmos com \texttt{lualatex} \ldots 
Mas, como fazemos isso?

Ora, a escrita do texto pode ser feita em qualquer \textbf{editor de texto} de 
sua preferência.
Pode-se usar um ``bloco de notas'' (para usuários de Windows), ou o ``evince'' (para
usuários de GNU/Linux), por exemplo.
Apenas deve-se lembrar em salvar o arquivo com extensão \texttt{.tex}.

Também é aconselhável manter o arquivo \texttt{main.tex} em algum diretório (pasta) 
nomeado adequadamente.
Isso se deve ao fato de que, ao compilarmos, arquivos secundários são gerados (
.aux, .log, etc.), o que pode gerar certa ``bagunça''.

Então, suponha que você tenha escrito um belo texto, cheio de equações matemáticas,
num arquivo denominado \texttt{main.tex}, salvo em um diretório por nome 
``\texttt{artigo\_tcc/}''.
\begin{marginfigure}
  \centering
  \includegraphics[width = 0.9\linewidth]{dir_lualatex.png}
  \caption{Arquivos gerados numa compilação simples}
\end{marginfigure}
Feito isso, abra o \textbf{terminal} (no Windows seria o ``Prompt de Comando'') 
dentro do diretório ``\texttt{artigo\_tcc/}'' e digite o seguinte comando:

\begin{codigo}{Usando o terminal para compilar}{\lapis}
  lualatex main.tex
\end{codigo}

Pelo menos três arquivos serão produzidos: um arquivo em \texttt{.pdf}, que é a 
saída desejada; um arquivo \texttt{.aux}, essencial em referências cruzadas, por 
exemplo; e, um arquivo \texttt{.log}, o qual é um registro detalhado de tudo o que
ocorreu na compilação, inclusive possíveis erros.
Pode aparecer mais arquivos durante a compilação!
Tudo dependerá de quão complexo é seu texto (se possui sumário; index; lista de tabela;
lista de figura; glossário; etc).
A Tabela~\ref{tab:aux} nos mostra alguns desses arquivos:

\begin{table}[!htbp]
  \centering
  \caption{Alguns dos possíveis arquivos gerados durante a compilação}  
  \label{tab:aux}
  \begin{tabular}{ll}
    \toprule
    \textbs{Extensão} & \textbs{Descrição}\\
    \midrule
    \texttt{.log} & \textrm{registro detalhado sobre a compilação, inclusive erros} \\
    \texttt{.aux} & \textrm{registra processos intermediários, como referências cruzadas} \\
    \texttt{.toc} & \textrm{serve para produção do índice} \\
    \texttt{.lof} & \textrm{serve para produção da lista de figuras} \\
    \texttt{.lot} & \textrm{serve para produção da lista de tabelas} \\
    \bottomrule
  \end{tabular}
\end{table}

Não tenha medo do terminal!
Ele é seu amigo!
Trabalhar usando ``linha de comando'', pelo terminal, é uma maneira de você 
comunicar-se com o computador de maneira rápida, direta e livre de distrações.
O comando \texttt{lualatex main.tex} simplesmente diz: ``Olha \ldots componha
tipograficamente meu texto que está no arquivo \texttt{main.tex}, usando o 
interpretador \hologo{LuaTeX}, no formato \hologo{LaTeX}''.

Mas, se você quiser mais funcionalidades na digitação (autocomplete, identação
automática, etc.) que um ``bloco de texto'' não oferece; bem como compilar sem 
usar o terminal, você precisará de um \textbf{editor de texto} especializado ou
uma IDE (\textit{Integrated Development Environment}).

\subsection{Editores para \LaTeX} %--------------------------------------------

\subsubsection{O \LaTeX\ é outro Word?} %--------------------------------------
\label{subsec:latex-word}

Um detalhe que precisamos ter em mente: a natureza dos editores para \LaTeX{}
é geralmente diferente dos editores de texto como \textit{Microsoft Word} ou 
\textit{LibreOffice Writer}.
Esses últimos editores são denominados WYSIWYG (acrônimo para a frase 
\textit{
  \textbf{W}hat \textbf{Y}ou \textbf{S}ee \textbf{I}s \textbf{W}hat \textbf{Y}ou \textbf{G}et
}).
\nota{
  ``WYSIWYG \ldots O termo é usado para classificar ferramentas de edição e 
  desenvolvimento que permitem visualizar, em tempo real, exatamente aquilo que 
  será publicado ou impresso'' (\hrefA{https://www.tecmundo.com.br/institucional/2057-o-que-e-wysiwyg-.htm}{TECMUNDO}, 
  acesso em 03/02/2022).\\
  Se você quiser saber como pronunciar corretamente essa sigla, acesse:
  \hrefA{https://youtu.be/GZuZxwIjh4I}{How to pronounce WYSIWYG}.
} 
Em uma tradução literal livre significa ``O que você vê é o que você tem'', ou 
seja, o que você vê na tela de seu editor, é o que aparecerá na impressão. 
No \LaTeX{} não é bem assim \ldots 
Você vê comandos misturados com texto normal, porém o resultado sairá diferente 
daquilo que estará vendo em seu editor! 
Ficará mais bonito o resultado, não se preocupe. =)

Poderíamos listar argumentos sobre a superioridade do \LaTeX{}, comparando-o ao 
\textit{Word} (por exemplo), mas isso não será feito. 
Cada um tem suas peculiaridades: vantajosas ou não. 
Tudo dependerá da finalidade do uso!
Particularmente, tenho encontrado no \LaTeX{} um ambiente próprio para escrita 
matemática de forma rápida, bonita e gratuita! 
Textos longos, cheios de capítulos, ou documentos personalizados --- como esse 
texto, exigiria um esforço e tempo tais, que nem passa por minha mente 
escrevê-los no Word (você já tentou criar um simples sumário no Word? 
Ou uma simples equação como esta: $\sum\limits_{n = 0}^{\infty} x^n = \frac{1}{1 - x}$, 
em menos de 10 segundos?)

Com isso em mente, é importante a aquisição de uma linguagem básica, mas, antes 
disso, vamos falar um pouco sobre os editores para \LaTeX.

\subsubsection{Indicações de editores para \LaTeX} %---------------------------

As possibilidades de editores são realmente vastas.
Portanto, comparar as funcionalidades dos principais editores é algo fundamental.
Você pode olhar uma comparação entre eles nesse link:

\begin{center}
  \textbf{
    \url{https://en.wikipedia.org/wiki/Comparison\_of\_TeX\_editors}
  }
\end{center}

Podemos usar editores mais \textbf{genéricos} ao \hologo{LaTeX} ou editores 
\textbf{dedicados}.

No que se refere aos editores mais genéricos, ou seja, que podem ser usados para 
outras linguagens de marcação (ou programação), podemos usar \textit{plugins}
para usarmos o \hologo{LaTeX} essa realidade muda. 
Veja a Tabela~\ref{tab:editores-genericos} para algumas sugestões.

\begin{margintable}
  \centering
  \caption{Editores Genéricos}
  \label{tab:editores-genericos}
  \medskip
  \begin{tabular}{ll}
    \toprule
      \textbs{Editor} & \textbs{\textit{Plugin}} \\
    \midrule
      \hrefA{https://www.gnu.org/software/emacs/emacs.html}{\sffamily GNU Emacs} & \href{https://www.gnu.org/software/auctex/}{\sffamily AUC \TeX}  \\                                       
      \hrefA{https://atom.io/}{\sffamily Atom}                                   & \href{https://atom.io/packages/atom-latex}{\sffamily atom-latex} \\
      \hrefA{https://neovim.io/}{\sffamily Neovim}                               & \href{https://github.com/lervag/vimtex}{\sffamily VimTeX}        \\
      \hrefA{https://code.visualstudio.com/}{\sffamily VSCode}                   & \href{https://marketplace.visualstudio.com/items?itemName=James-Yu.latex-workshop}{\sffamily \LaTeX\ Workshop}  \\  
    \bottomrule
  \end{tabular}  
\end{margintable}

Agora, no que se refere aos editores dedicados, a Tabela~\ref{tab:editores-dedicados} 
exibe algumas possibilidades.

\begin{table}[!h]
  \centering
  \caption{Alguns Editores Dedicados ao \LaTeX}
  \label{tab:editores-dedicados}
  \medskip
  \begin{tabular}{ll}
    \toprule
    \multicolumn{1}{p{2cm}}{\centering\textbs{Editor}}      & \multicolumn{1}{p{9.5cm}}{\centering\textbs{Algumas características}}\\
    \midrule
      \hrefB{https://www.texstudio.org/}{\textbs{\TeX studio}}       & \multicolumn{1}{p{9.5cm}}{\textrm{Destaca-se por ser multiplataforma e com muitos recursos.}}\\
      \hrefB{https://www.texniccenter.org/}{\textbs{\TeX nicCenter}} & \multicolumn{1}{p{9.5cm}}{\textrm{É voltado para usuários de Windows. Integra-se muito naturalmente ao sistema (e ainda indica o Sumatra para visualização do PDF).}}\\
      \hrefB{https://www.lyx.org/}{\textbs{LyX}}                     & \multicolumn{1}{p{9.5cm}}{\textrm{Esse é um editor diferente dos citados anteriormente; pois renderiza as equações e estruturas do texto de forma quase imediata. É o mais próximo, para \LaTeX, de editores WYSIWYG.}}\\
    \bottomrule
  \end{tabular}
\end{table}


Obviamente, para o iniciante, é indicado um editor dedicado ao \hologo{LaTeX}.

Particularmente, uso o VSCode com o \textit{plugin} do \LaTeX{}~\textit{Workshop}.
Esse \textit{plugin} faz o processo de compilação de uma forma um pouco 
diferente: ele usa uma ferramenta chamada \textbf{latexmk} que nada mais é do que 
um \textit{script} que automatiza muitos processos.
Por exemplo, quando o documento é complexo; com referências cruzadas; sumário; 
glossário; lista de tabelas; lista de figuras; etc. é necessário mais de um processo
de compilação. 
Esse \textit{plugin} faz isso tudo parecer mágica.
Ele vem, por padrão, configurado para \pdflatex, mas pode ser modificado
o interpretador de maneira bem simples.

Uma ferramenta parecida com o \texttt{latexmk}, mas que precisa de diretivas (
você precisa ``dizer'' para ele os passos que deseja executar) é o \textbf{arara}.
Ele foi desenvolvido por um brasileiro (Paulo Cereda) e possui ferramentas de 
\begin{marginfigure}
  \centering
  \href
  {
    https://gitlab.com/islandoftex/arara
  }
  {
    \includegraphics[width = \linewidth]{logo_arara.pdf}
  }
\end{marginfigure}
limpeza de arquivos auxiliares; rapidez na segunda compilação; produção de arquivo 
\textit{draft} (um rascunho que não gera o conteúdo pdf, mas apenas verifica a 
sintaxe do documento -- o que torna as coisas muito mais rápidas); etc.
Falaremos sobre ele no Apêndice~\ref{apend-A:arara}, pois vale muito a pena 
conhecer essa ferramenta!

\section{Overleaf: um editor \textit{online} notável} %------------------
\label{sec:overleaf}

Como nosso minicurso é \textit{online} e possui tão curto tempo, não seria 
possível um auxílio na instalação de uma distribuição \TeX{}, editor dedicado, 
nem tampouco um visualizador de PDF de cada um de vocês, localmente.

Usaremos, então, uma plataforma que reúne tudo isso de forma \textit{online} e 
gratuita (para aquilo que desejamos): \Overleaf.
\nota{
  Exitem outros editores online que poderíamos usar:
  \hrefA{https://papeeria.com/}{Papeeria} ou
  \hrefA{https://www.authorea.com/}{Authorea}, por exemplo.
}
Vamos conhecer alguns recursos disponíveis do \Overleaf, antes de iniciarmos a
aquisição da linguagem \hologo{LaTeX}.

O primeiro passo é acessar o \textit{link} e fazer um registro (\textit{Register}) 
de uma conta (é possível fazer o \textit{login} com uma conta Google), logando na
plataforma \textit{online} (Veja Figura~\ref{fig:overleaf_inicio}).

\begin{figure}[!htbp]
  \centering
  \href
  {
    https://www.overleaf.com/
  }
  {
    \includegraphics[width = 0.8\linewidth]{overleaf_inicio}
  }
  \caption{\url{https://www.overleaf.com/}}
  \label{fig:overleaf_inicio}
\end{figure}

Ao entrar no \Overleaf, aparecrá uma tela como é mostrado na Figura~\ref{fig:overleaf_panorama-EN}.

\begin{figure}[!h]
  \centering
  \includegraphics[width = 0.8\linewidth]{overleaf_panorama-EN}
  \caption{Visão inicial do Overleaf}
  \label{fig:overleaf_panorama-EN}
\end{figure}

Vamos explicar o que significa cada nicho destacado:

\begin{enumerate}
  \item Aqui é onde iniciaremos um \textbf{Novo Projeto} (\textit{New project}),
        que poderá ser em Branco; ou de algum modelo que o próprio \Overleaf{} 
        disponibiliza; ou do GitHub; etc.;
  \item Nesse nicho aparecerão todos os seus projetos (no meu caso, tenho 2 em 
        andamento);
  \item Você pode modificar o idioma por aqui;
  \item Também é possível fazermos a modificação do idioma por aqui;
  \item Você pode sair do \Overleaf{} clicando nesse botão e, em seguida, 
        \textsf{Sair} (\textit{LogOut})
\end{enumerate}

Escolha o item ``3'' ou ``4'' para modificar o idioma para \textsf{Portugues}.
Em seguida, abra um projeto em branco, seguindo o caminho:

\begin{center}
  \textsf{Novo Projeto} $\quad \longrightarrow \quad$ \textsf{Projeto em Branco}
\end{center}

Aparecerá uma janela para que você coloque o título do projeto.
Escreva:

\tcboxC{\sffamily \bfseries minicurso-EMAT\_LaTeX}

Entraremos naquilo que vamos denominar {\sffamily Área de Trabalho do Overleaf}
(algo como a Figura~\ref{fig:overleaf_desktop}).

\begin{figure}[!htbp]
  \centering
  \includegraphics[width = \linewidth]{overleaf_area-trabalho}
  \caption{Visão da Área de Trabalho dos Projetos, no Overleaf}
  \label{fig:overleaf_desktop}
\end{figure}

Vamos modificar o compilador (lembrem-se que usaremos o \lualatex; não o 
\pdflatex{}, que é o padrão do \Overleaf), mas, antes, vamos conhecer 
um pouco essa área de trabalho:

\begin{enumerate}
  \item Digitamos os códigos nesse espaço! 
        Aqui é onde escreveremos a lingugem do \hologo{LaTeX}.
        Notem que o \Overleaf{} já usou o seu nome e o nome do projeto para 
        preencher algumas coisas nessa linguagem;
  \item A saída do \texttt{pdf} é mostrada aqui.
        A renderização é automática, o que facilita muito o aprendizado para 
        iniciantes;
  \item Nesse espaço modificamos muita coisa no \Overleaf{}, especificamente, 
        modificamos algumas congigurações, inclusive o \textsf{compilador}.
        Também podemos fazer o \textit{download} do PDF ou do Código por aqui.
  \item No botão \textit{Recompile} existem muitas opções para renderizar o 
        documento.
  \item É aqui que você pode configurar para compilação automática; ou produzir
        um documento \textit{draft} (rascunho); ou para checagem das sintaxes;
  \item Por fim, aqui mostra mensagens de erros ou alertas. 
        Inclusive, há três mensagens de erros por lá! 
        Veremos que a mensagem, na realidade, resume-se a um único problema: 
        Ao aproveitar o título de nosso projeto e escrevê-lo no título do 
        documento, o \Overleaf{} usou um \textbf{caractere especial} no modo 
        \textsf{Texto}, mas que é exclusivamente reservado ao 
        \textsf{Modo Matemático} \emoji{man-shrugging-light-skin-tone}.
\end{enumerate}

\begin{atencao}{Atenção!}{\exclamacao}
  Em \textsf{Menu} (ver item 3 da Figura~\ref{fig:overleaf_desktop}), altere o
  compilador modificando na seção \textit{Settings}, item \textit{Compiler}, de
  \textsf{pdfLaTeX} para \textsf{LuaLaTeX}.
\end{atencao}

O nosso fluxo de trabalho será:

\begin{enumerate}[\ding{226}]
  \item \textbf{\textsf{escrever}} o texto e os códigos na área adequada;
  \item \textbf{\textsf{compilar}} (compor tipograficamente) o documento;
  \item \textbf{\textsf{depurar}} possíveis erros;
  \item \textbf{\textsf{admirar}} o resultado \emoji{relieved-face}.
\end{enumerate}

Algumas outras peculiaridades do \Overleaf{} veremos quando estivermos praticando
a linguagem do \LaTeX{} nos exercíos!
Então, agora, resta-nos aprender os rudimentos dessa linguagem.

\newpage