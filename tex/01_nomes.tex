\section{A Seiva da Árvore \ldots}\label{sec:seiva} %==========================

\begin{margintable}\vspace{.8in}\footnotesize
  \begin{tabularx}{\marginparwidth}{|X}
    Seção~\ref{sec:seiva}. A seiva da árvore\\
    Seção~\ref{sec:seiva}. A seiva da árvore\\
    Seção~\ref{sec:seiva}. A seiva da árvore\\
  \end{tabularx}
\end{margintable}

Lembro-me da primeira vez que vi o nome \LaTeX\ \ldots

Foi em uma chamada de um minicurso de alguma "semana de matemática" da 
universidade que fiz graduação.
O título era algo assim: "Introdução ao \LaTeX".

Eu estava a um semestre de concluir a graduação e pensei: 
\textit{
  "Rapaz \ldots acho que eles erraram esse minicurso. 
  Pra quê estudar látex em Matemática? 
  Isso está mais para Geografia."
}

\begin{marginfigure}
  \centering
  \href
  {
    https://www.globo.com/GloboRural/0,6993,EEC1703411-1935,00.html
  }
  {
    \includegraphics[width = 0.75\linewidth]{seringueira.jpg}
  }
  \caption{Isso é látex, não \LaTeX}
\end{marginfigure}

Sim! 
Eu pensei que estavam falando daquela substância espessa e branca que sai de 
algumas plantas (seringueira, por exemplo)!

Então, vamos deixar as coisas claras: não estamos falando de látex, mas de \LaTeX.

\begin{aviso}
  Aliás, como se pronuncia a palavra \LaTeX?\\
  Existem, pelo menos, duas maneiras de pronunciarmos, corretamente, essa palavra:
  "\textit{LeiTéc}" ou "\textit{LaTéc}", ou seja, o som de \TeX{} ("Téc") é o 
  mesmo que na palavra "\textit{tec}nologia". 
  Particularmente, adoto a segunda opção.
  Evite, por amor a Deus, falar "\textit{Látecks}".
\end{aviso}

Falando nisso, essa palavra \TeX{} não está destacada de forma aleatória! 
Ela foi idealizada como sendo a junção de três letras gregas: $\tau\epsilon\chi$.
\sidenote{%
  \footnotesize $\tau$ (tau), $\epsilon$ (épsilon) e $\chi$ (chi)
}
Esse núcleo grego gera palavras como \textit{arte} ($\small\tau\epsilon\chi\nu\eta$) 
ou mesmo \textit{tecnologia} ($\small\tau\epsilon\chi\nu o \lambda o \gamma\iota\alpha$).
Daí vem o espírito da palavra \TeX: unir uma arte (a tipografia) com a 
tecnologia (programação) para produzir documentos com uma beleza realmente ímpar.

Na realidade, para falarmos sobre \LaTeX, precisamos tangenciar o \TeX.

\subsection{Diferenciando os nomes} %------------------------------------------

Tudo começou quando \href{https://pt.wikipedia.org/wiki/Donald_Knuth}{Donald Ervin Knuth} 
queria qualidade nos elementos tipográficos de seus livros, principalmente na 
escrita matemática. 
Ele então criou, em meados da década dos anos 70, um sofisticado programa para a 
composição tipográfica de textos científicos e uma alternativa quase necessária 
para edição de textos com conteúdo matemático. 
Nasceu o \TeX. 
Em suas próprias palavras:

\begin{cita}
  \TeX{} é destinado para a criação de belos livros - e, especialmente, para os 
  livros que contêm grande quantidade de matemática.
\end{cita}

Todavia, ao que parece, essa linguagem de programação não era tão acessível 
\sout{a meros mortais como nós}, por conter diversos parâmetros relativos ao 
formato final do texto \ldots
Bom, vamos falar a verdade: dava muito trabalho para digitar o que você queria 
simplificar.
Isso porque o \TeX{} era formado por objetos denominados "primitivos".
O próprio Knuth criou um conjunto de macros\nota{
  Uma \textit{macro} (abreviação para macroinstrução), em ciência da computação, 
  é uma regra ou padrão que especifica como uma certa “sequência de entrada” 
  deve ser mapeada para uma substituição de “sequência de saída” de acordo com 
  um procedimento definido.
}, ou seja, um mapeamento de sequências de comandos primitivos, frequentemente 
utilizados, para simplificar a escrita de comandos \TeX{} em seus livros.
A esse conjunto de macros ele deu o nome de \textit{Plain \TeX}.

Obviamente, o \textit{Plain \TeX} é um conjunto de macros bem simples e que 
poderia ser expandido.
E foi isso que aconteceu.

\subsection{O surgimento do \LaTeX} %------------------------------------------

\href{https://pt.wikipedia.org/wiki/Leslie_Lamport}{Leslie B. Lamport}, facilitou 
nossa vida! 
Por volta dos anos 80, ele criou um conjunto de macros para o \TeX, muito bem 
estruturado, e com ideias interessantes (classes e pacotes, por exemplo)que 
somaram substanciamente à causa do \TeX{}, formando assim o \LaTeX.
Inclusive, o "La", de \LaTeX{}; vem do "La", de \textit{\textbf{La}mport}.\sidenote{
  $\text{\LaTeX} = \text{\textbf{La}mport} + \text{\TeX}$
}

Logo, nunca se esqueça disso:

\begin{center}
  \Ovalbox{
    \large  \textsf{O} \LaTeX\ \textsf{veio para facilitar sua vida}!
  }
\end{center}

\begin{figure}[!ht]
  \centering
  \includegraphics[width = 0.8\linewidth]{lamport-knuth}
  \caption{Os culpados!}
\end{figure}

\subsection{Mais e mais nomes} %-----------------------------------------------

Bom \ldots já sabemos que  \TeX{} e \LaTeX{} são coisas diferentes, entretanto,
interrelacionadas: o segundo é a forma mais utilizada, atualmente, para 
interagir com o primeiro.

Nesse ponto, seria interessante diferenciarmos \textit{engines} (motores/interpretadotes) 
de \textit{formats} (formatos).

O \TeX\ é um interpretador (\textit{engine}); o \LaTeX, um formato (\textit{format}).

\nota{
  Um outro formato que vimos é o \textit{Plain \TeX}; e, atualmente, um formato 
  que se destaca é o \href{https://en.wikipedia.org/wiki/ConTeXt}{\textcolor{azulUFRB}{Con\TeX t}}.
}

Os \textit{interpretadores} são os arquivos binários executáveis (ou seja, o 
programa em si); já os \textit{formatos} são macros (comandos ou instruções), 
baseadas em \TeX, que usamos para escrever nossos documentos (a grosso modo, são
linguagens ou atalhos para sequências de comandos ou estruturas em \TeX).

Não é difícil perceber que o interpretador \TeX\ é bem antigo e que outros tenham
surgido ao longo desses anos.
De fato, à época do \TeX, nem existia ainda arquivos com extensão  \texttt{.pdf} 
-- a saída dos documentos era em DVI.
Quando surgiu o PDF, um interpretador ficou bem conhecido: pdf\TeX{} -- cuja saída
poderia ser em DVI que poderia ser convertida em PDF. 
O pdf\TeX{} talvez seja o mais usado dentre os interpretadores que vamos falar, 
pois já vem selecionado, por padrão, em muitos editores dedicados ao \LaTeX\mn{
  falaremos sobre editores para \LaTeX\ mais abaixo.
}; as pessoas, simplesmente, não mudam a configuração padrão se realmente não 
for necessário.
Nosso sistema relaciona o \LaTeX\ com o pdf\LaTeX\ usando um "atalho", denominado
pdf\LaTeX.

Portanto, o pdf\LaTeX\ (que geralmente é o nome que aparecerá em nossos editores) 
é o \textit{interpretador} pdf\TeX\ com o \textit{formato} \LaTeX. 

Atualmente, dois interpretadores que se destacam são Xe\TeX\ e Lua\TeX.
Além de serem mais rápidos, trazem implementações notáveis como seleção de fontes 
do próprio sistema (o pdf\TeX\ não faz isso) ou interação com a linguagem de 
programação (brasileira) Lua.
A saída de cada um deles é o PDF, diretamente.

\begin{marginfigure}
  \centering
  \href
  {
    https://www.lua.org/portugues.html
  }
  {
    \includegraphics[width = 0.6\linewidth]{logo_lua.png}
  }
  \caption{Poderosa linguagem de programação (clique na imagem para saber mais)}
\end{marginfigure}
 
Em nossos sistemas, ao usarmos Xe\TeX\ com \LaTeX\ usamos o atalho Xe\LaTeX.
Da mesma forma, Lua\TeX\ com \LaTeX é simplificado por Lua\LaTeX.

Geralmente chamamos esses "atalhos" de \textbf{compiladores}.
Veja a Tabela~\ref{tab:compiladores}.

\begin{table}[!htbp]
  \centering
  \begin{tabular}{llc}
    \toprule
      \textbf{Compilador} & \textbf{Interpretador} & \textbf{Formato}\\
    \midrule
      \texttt{pdflatex} & pdf\TeX & \LaTeX\\
      \texttt{xelatex}  & Xe\TeX  & \LaTeX\\
      \texttt{lualatex} & Lua\TeX & \LaTeX\\
    \bottomrule
  \end{tabular}
  \caption{Dando "nome aos bois"}
  \label{tab:compiladores}
\end{table}

Obviamente, nesse minicurso, estamos interessados apenas no formato \LaTeX.
Também, por mera preferência, usaremos como interpretador o Lua\TeX.
Portanto, "compilaremos" (ou seja, a composição tipográfica) nossos arquivos 
com \texttt{lualatex}.

\begin{aviso}
  Isso pode mudar algumas coisas para quem é acostumado a usar \texttt{pdflatex}, 
  mas explicaremos as diferenças ao longo desse minicurso.  
\end{aviso}

Obtamos por usar \lualatex, pois é o sucessor natural do \texttt{pdflatex}. \sidenote{
  \footnotesize
  Confira essa informação no site \href{http://www.luatex.org/}{\textbf{http://www.luatex.org/}}
}

Para encerramos essa subseção, é importante destacar a estabilidade do \LaTeX.
Basicamente temos duas versões: uma antes de 1993, a saber \LaTeX~2.09; e,
a de 1994 até HOJE, a saber \LaTeX$2\varepsilon$.
Isso é interessante, pois os códigos são praticamente preservados ao longo do 
tempo: você poderia rodar um código em \LaTeX\ de 30 anos atrás sem muitos 
problemas.

Todavia, quando falamos em \LaTeX{}, hoje, estamos nos referindo às 
funcionalidades trazidas na versão \LaTeX$2\varepsilon$.

As atualizações são constantes, mas sem muitas mudanças estruturais.
Todavia, está em desenvolvimento uma terceira versão do \LaTeX, a saber, \LaTeX3.
Você pode saber mais no link abaixo:

\begin{figure}[!htbp]
  \centering
  \includegraphics[width = 0.7\linewidth]{logo_latex}
  \caption{\href{https://www.latex-project.org/latex3/}{\textcolor{azulUFRB}{https://www.latex-project.org/latex3/}}}
\end{figure}

 

